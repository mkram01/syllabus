\documentclass[]{book}
\usepackage{lmodern}
\usepackage{amssymb,amsmath}
\usepackage{ifxetex,ifluatex}
\usepackage{fixltx2e} % provides \textsubscript
\ifnum 0\ifxetex 1\fi\ifluatex 1\fi=0 % if pdftex
  \usepackage[T1]{fontenc}
  \usepackage[utf8]{inputenc}
\else % if luatex or xelatex
  \ifxetex
    \usepackage{mathspec}
  \else
    \usepackage{fontspec}
  \fi
  \defaultfontfeatures{Ligatures=TeX,Scale=MatchLowercase}
\fi
% use upquote if available, for straight quotes in verbatim environments
\IfFileExists{upquote.sty}{\usepackage{upquote}}{}
% use microtype if available
\IfFileExists{microtype.sty}{%
\usepackage{microtype}
\UseMicrotypeSet[protrusion]{basicmath} % disable protrusion for tt fonts
}{}
\usepackage[margin=1in]{geometry}
\usepackage{hyperref}
\hypersetup{unicode=true,
            pdftitle={SOC 4015 \& 5050: Quantitative Analysis},
            pdfauthor={Christopher Prener, Ph.D.},
            pdfborder={0 0 0},
            breaklinks=true}
\urlstyle{same}  % don't use monospace font for urls
\usepackage{natbib}
\bibliographystyle{apalike}
\usepackage{longtable,booktabs}
\usepackage{graphicx,grffile}
\makeatletter
\def\maxwidth{\ifdim\Gin@nat@width>\linewidth\linewidth\else\Gin@nat@width\fi}
\def\maxheight{\ifdim\Gin@nat@height>\textheight\textheight\else\Gin@nat@height\fi}
\makeatother
% Scale images if necessary, so that they will not overflow the page
% margins by default, and it is still possible to overwrite the defaults
% using explicit options in \includegraphics[width, height, ...]{}
\setkeys{Gin}{width=\maxwidth,height=\maxheight,keepaspectratio}
\IfFileExists{parskip.sty}{%
\usepackage{parskip}
}{% else
\setlength{\parindent}{0pt}
\setlength{\parskip}{6pt plus 2pt minus 1pt}
}
\setlength{\emergencystretch}{3em}  % prevent overfull lines
\providecommand{\tightlist}{%
  \setlength{\itemsep}{0pt}\setlength{\parskip}{0pt}}
\setcounter{secnumdepth}{5}
% Redefines (sub)paragraphs to behave more like sections
\ifx\paragraph\undefined\else
\let\oldparagraph\paragraph
\renewcommand{\paragraph}[1]{\oldparagraph{#1}\mbox{}}
\fi
\ifx\subparagraph\undefined\else
\let\oldsubparagraph\subparagraph
\renewcommand{\subparagraph}[1]{\oldsubparagraph{#1}\mbox{}}
\fi

%%% Use protect on footnotes to avoid problems with footnotes in titles
\let\rmarkdownfootnote\footnote%
\def\footnote{\protect\rmarkdownfootnote}

%%% Change title format to be more compact
\usepackage{titling}

% Create subtitle command for use in maketitle
\newcommand{\subtitle}[1]{
  \posttitle{
    \begin{center}\large#1\end{center}
    }
}

\setlength{\droptitle}{-2em}
  \title{SOC 4015 \& 5050: Quantitative Analysis}
  \pretitle{\vspace{\droptitle}\centering\huge}
  \posttitle{\par}
  \author{Christopher Prener, Ph.D.}
  \preauthor{\centering\large\emph}
  \postauthor{\par}
  \predate{\centering\large\emph}
  \postdate{\par}
  \date{2018-05-02}

\usepackage{booktabs}
\usepackage{amsthm}
\makeatletter
\def\thm@space@setup{%
  \thm@preskip=8pt plus 2pt minus 4pt
  \thm@postskip=\thm@preskip
}
\makeatother

\newenvironment{rmdblock}[1]
  {\begin{shaded*}
  \begin{itemize}
  \renewcommand{\labelitemi}{
    \raisebox{-.7\height}[0pt][0pt]{
      {\setkeys{Gin}{width=3em,keepaspectratio}\includegraphics{images/#1}}
    }
  }
  \item
  }
  {
  \end{itemize}
  \end{shaded*}
  }
\newenvironment{rmdnote}
  {\begin{rmdblock}{note}}
  {\end{rmdblock}}
\newenvironment{rmdtip}
  {\begin{rmdblock}{tip}}
  {\end{rmdblock}}
\newenvironment{rmdwarning}
  {\begin{rmdblock}{warning}}
  {\end{rmdblock}}

\usepackage{amsthm}
\newtheorem{theorem}{Theorem}[chapter]
\newtheorem{lemma}{Lemma}[chapter]
\theoremstyle{definition}
\newtheorem{definition}{Definition}[chapter]
\newtheorem{corollary}{Corollary}[chapter]
\newtheorem{proposition}{Proposition}[chapter]
\theoremstyle{definition}
\newtheorem{example}{Example}[chapter]
\theoremstyle{definition}
\newtheorem{exercise}{Exercise}[chapter]
\theoremstyle{remark}
\newtheorem*{remark}{Remark}
\newtheorem*{solution}{Solution}
\begin{document}
\maketitle

{
\setcounter{tocdepth}{1}
\tableofcontents
}
\chapter*{Basics}\label{basics}
\addcontentsline{toc}{chapter}{Basics}

\begin{rmdwarning}
This is the \textbf{draft} syllabus for \textbf{Fall 2018}. Changes
should expected before the release of the syllabus prior to the first
day of class. This will occur in mid August.
\end{rmdwarning}

\subsection*{Course Meetings}\label{course-meetings}
\addcontentsline{toc}{subsection}{Course Meetings}

\emph{When:} Mondays, 4:15pm to 7:00pm

\emph{Where:} 3600 Morrissey (GeoSRI Lab)

\subsection*{Course Website}\label{course-website}
\addcontentsline{toc}{subsection}{Course Website}

\url{https://slu-soc5050.github.io}

\subsection*{Chris's Information}\label{chriss-information}
\addcontentsline{toc}{subsection}{Chris's Information}

\emph{Office:} 1918 Morrissey Hall

\emph{Email:}
\href{mailto:chris.prener@slu.edu}{\nolinkurl{chris.prener@slu.edu}}

\emph{GitHub:} \texttt{@chris-prener}

\emph{Slack:} \texttt{@chris}

\emph{Office Hours:}

\begin{itemize}
\item
  Mondays, 7:00pm to 7:30pm in 3600 Morrissey (GeoSRI Lab)
\item
  Wednesdays, 10:00am to 12:00pm in 3600 Morrissey (GeoSRI Lab)
\end{itemize}

\section*{License}\label{license}
\addcontentsline{toc}{section}{License}

Copyright © 2016-2018 \href{https://chris-prener.github.io}{Christopher
G. Prener}

This work is licensed under a Creative Commons Attribution-ShareAlike
4.0 International License.

\part{Syllabus}\label{part-syllabus}

\chapter{Course Introduction}\label{course-introduction}

\begin{quote}
Figures often beguile me, particularly when I have the arranging of them
myself; in which case the remark attributed to Disraeli would often
apply with justice and force: `There are three kinds of lies: lies,
damned lies, and statistics'.
\end{quote}

\textbf{Mark Twain (1906)}

This course provides an introduction to applied statistical analysis for
both undergraduate and graduate students with an emphasis placed on
statistical techniques that are most common in the sociological
literature. The statistical techniques introduced include measures of
central tendency and dispersion as well as measures of bi-variate
association. Multivariate statistical analyses are also introduced along
with essential skills for cleaning data, creating plots, and reporting
results. While the examples may be specific to the social sciences, the
theories and skills that are covered are broadly applicable across
academic disciplines.

\section{Two Courses, One Goal}\label{two-courses-one-goal}

Students will quickly noticed that this course has two numbers. SOC 4015
is the undergraduate section, and SOC 5050 is the graduate section. This
quickly leads to anxiety for some students, who worry they have signed
up for the wrong class (ocasionally this is not misplaced anxiety!) or
who worry that they are taking a class that is not appropriate for their
skill level. This class is designed for social science students with
little to no background in statistics \texttt{R}, and scientific
computing more generally. For those students, the level is largely
irrelevent - undergraduate and graduate students who have not been
exposed to these ideas need to cover the same material.

Graduate students who take this class will have to do some additional
work - the final project is more rigorus than the project that
undergraduates will complete. Otherwise, the course is the same because
what content students need is largely the same as well.

\section{Course Objectives}\label{course-objectives}

This course has four intertwined objectives. After completing the
course, students will be able to:

\begin{enumerate}
\def\labelenumi{\arabic{enumi}.}
\item
  \emph{Fundamentals of inferential statistics}: Describe the use of
  various statistical tests, their requirements and assumptions, and
  their interpretation; execute these tests both by hand and
  programmatically using \texttt{R}.
\item
  \emph{Fundamentals of data analysis}: Perform basic data cleaning and
  analysis tasks programmatically using \texttt{R} in ways that support
  high quality documentation and replication.
\item
  \emph{Fundamentals of data visualization}: Create and present
  publication quality plots programmatically using \texttt{R} and
  \texttt{ggplot2}.
\item
  \emph{Quantitative research synthesis}: Plan, implement (using
  \texttt{R}), and present (using LaTeX and the presentation software of
  your choice) a research project that uses linear regression to answer
  a research question.
\end{enumerate}

\section{Core Resources}\label{core-resources}

There are two core documents and resources for this course. This
\textbf{Syllabus} sets out core expectations and policies for the course
- i.e.~what is \emph{required} for this course.. It includes a
\textbf{Reading List} that contains topics, readings (both required and
optional), and assignment due dates for each week. Once the semester
starts, these documents will only be updated if a schedule change is
necessary.

In addition to these documents, regular updates will be provided on the
\href{https://slu-soc5050.github.io}{\textbf{course website}}. Each
lecture will have a corresponding page on the site that includes links
to handouts, YouTube videos, sample code, and additional descriptions of
concepts covered in class. If bugs or issues arise, they will be
documented along with solutions here as well. Please check the website
regularly for updates and new content.

\section{Readings}\label{readings}

There are three books required for this course with an optional fourth
book. Each book has been selected to correspond with one or more of the
course objectives. The books are:

\begin{enumerate}
\def\labelenumi{\arabic{enumi}.}
\tightlist
\item
  Diez, David M., Christopher D Barr, and Mine Cetinkaya-Rundel-Runde.
  2015. \emph{OpenIntro Statistics}. 3rd edition. OpenIntro.

  \begin{itemize}
  \tightlist
  \item
    This book is \textbf{not} available from the Bookstore. You can
    \href{https://www.openintro.org/stat/textbook.php}{download a free
    copy} or purchase a physical copy from Amazon
    (\href{https://www.amazon.com/dp/1943450048/}{black \& white} or
    \href{https://www.amazon.com/dp/1943450056/}{color}).
  \end{itemize}
\item
  Prener, Christopher. 2018. \emph{Sociospatial Data Science}.

  \begin{itemize}
  \tightlist
  \item
    This book is \textbf{not} available from the Bookstore. You can
    access it as a webbook and download it as a \texttt{.pdf}
    \href{https://chris-prener.github.io/SSDSBook/}{here}.
  \end{itemize}
\item
  Wheelan, Charles. 2014. \emph{Naked Statistics: Stripping the Dread
  from the Data}. New York, NY: W. W. Norton \& Company.

  \begin{itemize}
  \tightlist
  \item
    This book can be purchased in the bookstore or online. Ebook
    versions are available.
  \end{itemize}
\item
  Wickham, Hadley and Garrett Grolemund. 2017. \emph{R for Data
  Science}. O'Reily Media: Sebastopol, CA.

  \begin{itemize}
  \tightlist
  \item
    This book book can be purchased in the bookstore, online, or
    accessed for free \href{http://r4ds.had.co.nz}{as a webbook}.
  \end{itemize}
\end{enumerate}

I do not require students to buy physical copies of texts. You are free
to select a means for accessing these texts that meets your budget and
learning style. If ebook editions (e.g.~Kindle, iBooks, \texttt{pdf},
etc.) of texts are available, they are acceptable for this course. All
texts should be obtained in the edition noted above.

All readings are listed on the \textbf{Reading List} and should be
completed before the course meeting on the week in which they are
assigned. Full text versions of most readings not found in the books
assigned for the course can be obtained using the library's
\href{http://eres.slu.edu/eres/coursepass.aspx?cid=4487}{Electronic
Reserves} system. The password for the Electric Reserves will be posted
on Slack at the beginning of the semester.

\section{Services}\label{services}

Over the course of the semester, we'll use three web-based services.
Each of these will require you to create an account with a username and
password. GitHub will require you to enable
\href{https://en.wikipedia.org/wiki/Multi-factor_authentication}{two-factor
authentication} as well, and you should also enable this for Slack. I
strongly recommend using a
\href{https://lifehacker.com/5529133/five-best-password-managers}{password
manager}.

All of these services have free tiers as well as premium features that
require a monthly subscription. None of these premium features are
required for this course - what you can access for free is all the
functionality you will need!

\subsection{GitHub}\label{github}

The majority of course content (sample code, documentation, and
assignments) for this course will be made available using
\textbf{\href{http://www.github.com}{GitHub}}. GitHub is a website used
by programmers, data analysts, and researchers to share computer code
and projects. GitHub will also be used for assignment submission and
feedback. In addition to providing us with platform for hosting course
content, using GitHub will give you experience in some of the techniques
that researchers use to conduct both open-source and collaborative
research. GitHub is free to use but does have some premium features,
which students can access for free through their
\href{https://education.github.com/pack/}{Student Developer program}. As
I noted above, these premium features \emph{are not required} for this
course but are worth knowing about if you decide to continue using
GitHub.

\subsection{Slack}\label{slack}

We will be utilizing the communication service
\textbf{\href{https://slack.com}{Slack}} to stay in touch. Slack allows
me to post announcements and updates about the course that you will
receive alerts to. Any changes to our course GitHub repositories will
also be posted there automatically. Slack will also provide us with a
space to host virtual office hours. This allows students to monitor the
types of questions and issues that are arising, and learn from each
other's experiences. Slack can be accessed via a web browser or you can
optionally install mobile as well as
\href{https://slack.com/downloads/osx}{desktop applications} available
for both Windows and macOS.

\subsection{ShareLaTeX}\label{sharelatex}

Finally, we'll use a free web service called
\href{https://www.sharelatex.com}{ShareLaTeX} to create documents using
the markup language LaTeX. Students' final projects will need to
incorporate LaTeX deliverables. While LaTeX can be downloaded and
installed, learning LaTeX online takes away some of the initial
challenges with getting a LaTeX installation up and running. It will
also allow me to help trouble-shoot issues remotely.

In July 2017,
\href{https://www.sharelatex.com/blog/2017/07/20/sharelatex-joins-overleaf.html}{ShareLaTeX
joined forces with their primary competitor, Overleaf}. Eventually one
or both of these services may disappear, so there may be changes in how
we access LaTeX mid-semester.

\section{Software}\label{software}

There are two principle applications we'll be using this semester in
addition to the services listed previously:
\href{https://www.rstudio.com}{RStudio} and
\href{https://desktop.github.com}{GitHub Desktop}. Both of these are
open-source applications that can be downloaded and used without cost.
Both applications are available in our classroom, which you will have
24-hour access to throughout the semester.

\subsection{\texorpdfstring{\texttt{R} and
RStudio}{R and RStudio}}\label{r-and-rstudio}

The primary tool we will use for data manipulation and analysis is the
programming language \texttt{R}. \texttt{R} is open-source, freely
available, and highly extensible analysis environment. We'll use
\href{https://www.rstudio.com}{RStudio} as the ``front end'' for our
analyses. RStudio makes it easier to write \texttt{R} code and to
produce well documented analyses. Like the \texttt{R} programming
language itself, RStudio is freely available.

Regardless if you are going to use RStudio on your computer or in our
classroom, you have two options available for accessing the software:

\begin{enumerate}
\def\labelenumi{\arabic{enumi}.}
\tightlist
\item
  Download \texttt{R}, RStudio, and the necessary packages manually and
  manage your own installation of these tools.
\item
  Access \texttt{R} and RStudio via
  \href{https://www.docker.com}{Docker}, a tool for creating virtual
  computing enviornments.
\end{enumerate}

Students who are not sure whether they will use \texttt{R} past this
semester, or who are less comfortable with computers, are urged to
access \texttt{R} via Docker. Students who have more comfort with
computers and who already are \texttt{R} users or plan to continue using
\texttt{R} after the semester should consider managing their own
installation of these tools.

Detailed instructions are available for both options on the
\href{https://slu-soc5050.github.io/course-software/}{course website}.

\subsection{GitHub Desktop}\label{github-desktop}

You will need another free application called
\href{https://desktop.github.com}{GitHub Desktop}. This program allows
you to easily copy data from GitHub onto your computer. It also makes it
easy to upload files like labs and problem sets to GitHub. If you have
already used Git via the command line, you can continue to do so without
utilizing GitHub Desktop.

\chapter{Course Policies}\label{course-policies}

My priority is that class periods are productive learning experiences
for all students. In order to foster this type of productive
environment, I ask students to follow a few general policies and
expectations:

\begin{enumerate}
\def\labelenumi{\arabic{enumi}.}
\tightlist
\item
  Work each week to contribute to a positive, supportive, welcoming, and
  compassionate class enviornment.
\item
  Arrive to class on time and stay for the entire class period.
\item
  Silence \emph{all} electronic devices before entering the classroom.
\item
  Do not engage in side conversations. This is disrespectful to the
  speaker (whether me or a classmate), and can affect the ability of
  others in the class to learn.
\item
  Be respectful of your fellow classmates. Do not interrupt when someone
  is speaking, monopolize the conversation, or belittle the ideas or
  opinions of others.
\item
  Complete the assigned readings for each class in advance, and come
  prepared with discussion points and questions.
\end{enumerate}

The following sections contain additional details about specific course
policies related to attendance, participation, electronic device use,
student support, academic honesty, and Title IX.

\section{Compassionate Coursework}\label{compassionate-coursework}

\begin{quote}
Being around people who are different from us makes us more creative,
more diligent and harder-working
\end{quote}

\textbf{\href{https://www.scientificamerican.com/article/how-diversity-makes-us-smarter/}{Katherine
Phillips, 2014}}

 The goal of this course is not just to impart knowledge related to
statistics, data science, and programming, but to purposefully create an
enviornment where \textbf{all} students feel welcome and supported even
as they also feel challenged intellectually. This is especially
important in a STEM course, where stress levels can be generally high,
particularly for certain groups of students.

\subsection{Challenges in STEM
Coursework}\label{challenges-in-stem-coursework}

Statistics courses, and research methods coursework more generally, can
be stressful. This can be true for many reasons. For students who have
not had significant mathmatical coursework since High School, or have
not written computer code before, that could be enough to treat
statistics coursework with aprehension.

Within STEM coursework, there may also be other, more pernicious
patterns. Men may rate their abilities more strongly than women
(\href{https://www.physiology.org/doi/10.1152/advan.00085.2017}{Cooper
et al. 2018}) and students who do not identify as ``male'' or ``white''
may not feel like they fit into STEM settings
(\href{https://doi.org/10.1525/sp.2005.52.4.593}{Ong 2005}). This can
lead to ``imposter syndrome'', a feeling that academic gains are not the
result of a students' abilities and a fear that they will soon be
``found out''
(\href{http://genderandset.open.ac.uk/index.php/genderandset/article/view/435}{Lindemann
et al. 2016}). These feelings can make STEM coursework particularly
isolating for women and students of color
(\href{https://onlinelibrary.wiley.com/doi/pdf/10.1002/sce.20307}{Malone
and Barabino 2008},
\href{https://doi.org/10.17763/haer.81.2.t022245n7x4752v2}{Ong et al.
2011}).

While attrition rates out of STEM majors are subsantially similar for
men and women, there are generally fewer women in these programs to
begin with and so losses of female students are particularly concerning
(\href{https://www.aauw.org/aauw_check/pdf_download/show_pdf.php?file=why-so-few-research}{Hill
et al. 2010}). Attrition at the University level may have second order
effects for employers, for whom diversity is a critical piece of their
success
(\href{https://open.nytimes.com/why-having-a-diverse-team-will-make-your-products-better-c73e7518f677}{Akinnawonu
2017},
\href{https://www.tandfonline.com/doi/abs/10.5172/impp.2013.15.2.149}{Diaz-Garcia
et al. 2011},
\href{https://www.tandfonline.com/doi/abs/10.1111/ecge.12016}{Nathan and
Lee 2015}).

\subsection{A Compassionate Classroom}\label{a-compassionate-classroom}

Addressing these challenges, and producing a compassionate coursework
experience for every student, is \textbf{all} of our responsibility.
Taking this responsibility seriously is imperative not just for reasons
of academic retention and its future implications, but also because we
are called to do so by the
\href{https://www.slu.edu/about/catholic-jesuit-identity/mission.php}{University's
mission} both in our classrooms and in the wider world.

While I take a leading role in fostering a welcoming and supportive
enviornment, I need each and every student's help in making that
enviornment a reality. To that end, all students should familiarize
themselves with \href{https://www.contributor-covenant.org}{Contributor
Covenant's}
\href{https://www.contributor-covenant.org/version/1/4/code-of-conduct}{Code
of Conduct}, which is increasingly included in open source projects. The
Code of Conduct lays out expectations for how all students are expected
to conduct themselves.

I want to emphasize one piece here in the syllabus, which includes
concreate examples of things each student \emph{can} and \emph{should}
do to help create a compassionate class atmosphere:

\begin{quote}
Examples of behavior that contributes to creating a positive environment
include: using welcoming and inclusive language, being respectful of
differing viewpoints and experiences, gracefully accepting constructive
criticism, focusing on what is best for the {[}class{]}, {[}and{]}
showing empathy towards other community members
\end{quote}

The degree to which students are positively engaged with our class along
these lines will be reflected in participation grades given at the mid
and end points of the semester.

\subsection{Harrassment and Title IX}\label{harrassment-and-title-ix}

While I have every expectation that each member of the Saint Louis
University community is capable and willing to create a compassionate
coursework experience, I fully recognize that there may be instances
where students fall short of that expectation. Students should generally
be aware that:

\begin{quote}
Saint Louis University prohibits harassment because of sex, race, color,
religion, national origin, ancestry, disability, age, sexual
orientation, marital status, military status, veteran status, gender
expression/identity, genetic information, pregnancy, or any other
characteristics protected by law.
\end{quote}

All students should also familiarize themselves with
\href{http://www.slu.edu/general-counsel-home/office-of-institutional-equity-and-diversity}{Saint
Louis University's polices} on bias, discrimination, harassment, and
sexual misconduct. In particular, they should be aware of policies on
\href{https://www.slu.edu/general-counsel/institutional-equity-diversity/pdfs/harassment-policy.pdf}{harassment}
and
\href{https://www.slu.edu/about/safety/sexual-assault-resources.php}{sexual
misconduct}:

\begin{quote}
Saint Louis University and its faculty are committed to supporting our
students and seeking an environment that is free of bias,
discrimination, and harassment. If you have encountered any form of
sexual misconduct (e.g.~sexual assault, sexual harassment, stalking,
domestic or dating violence), we encourage you to report this to the
University. If you speak with a faculty member about an incident of
misconduct, that faculty member must notify SLU's Title IX Coordinator,
Anna R. Kratky (DuBourg Hall, Room 36;
\href{mailto:anna.kratky@slu.edu}{\nolinkurl{anna.kratky@slu.edu}});
314-977-3886) and share the basic facts of your experience with her. The
Title IX coordinator will then be available to assist you in
understanding all of your options and in connecting you with all
possible resources on and off campus.
\end{quote}

\begin{quote}
If you wish to speak with a confidential source, you may contact the
counselors at the University Counseling Center at 314-977-TALK.
\end{quote}

Instances of abusive, harassing, or otherwise unacceptable behavior
should be reported either directly to the instructor or to the
University Administration. Consistent with the above policies, I will
forward all reports of inappropriate conduct to the Title IX
Coordinator's office or to the Office of Diversity and Affirmative
Action. Please be aware that University policies may require me to
forward information about the identity of any students connected to the
disclosure.

Please also be aware that communications over various online services,
including (but not limited to) Slack, GitHub, ShareLaTeX, and Google
Apps, are covered by this policy.

\section{Attendance and
Participation}\label{attendance-and-participation}

Attendance and participation are important components of this course
since we only meet once a week. Students are expected to attend all
class sessions; missing even one class can create a significant
roadblock for many students. Making up missed classes are your
responsibility. Please let me know if you will be missing class so that
we can touch base about any assignments, and make sure to obtain notes
from a classmate. I do post slides to GitHub, but my slides are intended
only to serve as references. Please note that lectures and discussions
cannot be recorded by any means (e.g.~audio or video recordings, or
photographs) without my permission.

\section{Communication}\label{communication}

Slack and email are my preferred methods of communication.

I am on Slack during workday hours, though I may be ``away'' during
meetings and while I teach my other class. I also may have limited
availability on Tuesdays, a day that I am not typically on-campus. I
will also monitor Slack during weekday evening hours and will respond to
messages if I am able. Likewise, I dedicate time to email responses each
workday, meaning that my response time is typically within 24 hours
during the workweek. Please use your SLU email account when emailing me.

For both email and Slack, if you have not received a response from me
after 48 hours (or by end of business on Monday if you messaged me over
the weekend), please follow-up to ensure that your message did not get
lost in the shuffle.

All messages regarding course updates, assignments, and changes to the
class schedule, including cancellations, will be posted on the
\texttt{\_news} channel in Slack. Changes to the class schedule,
including cancellations, will also sent to your SLU email account. It is
imperative that you check both Slack and your SLU email account
regularly.

Please also ensure that all concerns or questions about your standing in
the course are directed to me immediately. Inquires from parents, SLU
staff members, and others will not be honored.

\section{Electronic Devices}\label{electronic-devices}

During class periods, students are asked to refrain from using
electronic devices (including cell phones) for activities not directly
related to the course. For this class, I expect students to limit their
use of electronic devices to accessing course software, readings, and
notes.

There is evidence that using electronic devices during lectures results
in decreased retention of course content
(\href{https://link.springer.com/article/10.1007/BF02940852}{Hembrooke
and Gay 2003}) and lower overall course performance
(\href{https://www.sciencedirect.com/science/article/pii/S0360131506001436}{Fried
2008}). Students who are not using a laptop but are in direct view of
another student's laptop also have decreased performance in courses
(\href{https://www.sciencedirect.com/science/article/pii/S0360131512002254}{Sana
et al. 2013}). Conversely, students who take notes the ``old fashioned
way'' have better performance on tests compared to students who take
notes on laptops
(\href{http://journals.sagepub.com/doi/abs/10.1177/0956797614524581}{Mueller
and Oppenheimer 2014}).

I therefore ask students to be conscious of how they are using their
devices, the ways such use impacts their own learning, and the effect
that it may have on others around them. I reserve the right to alter
this policy if electronic device use becomes problematic during the
semester.

\section{Student Support}\label{student-support}

If you meet the eligibility requirements for \textbf{academic
accommodations} through the Disability Services office (located within
the Student Success Center), you should arrange to discuss your needs
with me after the first class. All discussions of this nature are
treated confidentially, and I will make every effort to work with you to
come up with a plan for successfully completing the course requirements.
Please note that I will not provide accommodations to students who are
not working with Disability Services.

If you are a \textbf{student-athlete} who is in-season, you should
discuss your game schedule with me after the first class and share your
travel letter with me as soon as you have a copy. You are reminded that
games and tournaments are not excuses for failing to complete
assignments, and that NCAA rules prohibit student-athletes from missing
classes for practice. Low grades that jeopardize eligibility must be
addressed immediately by you, not by a coach or academic coordinator.

I also encourage you to take advantage of the \textbf{University Writing
Services (UWS) program}. Getting feedback benefits writers at all skill
levels and the quality of your writing will be reflected in assignment
grades. The UWS has trained writing consultants who can help you improve
the quality of your written work. UWS's consultants are available to
address everything from brainstorming and developing ideas to crafting
strong sentences and documenting sources.

\section{Academic Honesty}\label{academic-honesty}

All students should familiarize themselves with
\href{http://www.slu.edu/Documents/provost/academic_affairs/Academic\%20Integrity\%20Policy\%20FINAL\%20\%206-26-15.pd}{Saint
Louis University's policies} concerning cheating, plagiarism, and other
academically dishonest practices:

\begin{quote}
Academic integrity is honest, truthful and responsible conduct in all
academic endeavors. The mission of Saint Louis University is ``the
pursuit of truth for the greater glory of God and for the service of
humanity.'' Accordingly, all acts of falsehood demean and compromise the
corporate endeavors of teaching, research, health care, and community
service via which SLU embodies its mission. The University strives to
prepare students for lives of personal and professional integrity, and
therefore regards all breaches of academic integrity as matters of
serious concern.
\end{quote}

Any work that is taken from another student, copied from printed
material, or copied the internet without proper citation is expressly
prohibited. Note that this includes all computer code, narrative text,
and documentation written for class assignments - each student is
expected to author and de-bug their notebooks and accompanying files.

All relevant assignments should include in-text citations and references
formatted using the
\href{https://owl.english.purdue.edu/owl/resource/560/01/}{American
Psychological Association (APA)} style guidelines. Any student who is
found to have been academically dishonest in their work risks failing
both the assignment and this course.

\chapter{Assignments and Grading}\label{assignments-and-grading}

This section provides general details on the different types of
assignments for this course. It also contains policies for submitting
work, reciving feedback, and late work.

\section{Assignments}\label{assignments}

Your grade for this course will consist of a number of different
assignments on which points may be earned. Each category of assignment
is described below.

\subsection{Attendance and
Participation}\label{attendance-and-participation-1}

\begin{rmdtip}
Attendance and participation are worth \textbf{10\%} of your final
grade.
\end{rmdtip}

Both attendance and participation are critically important aspects of
this class. The class participation grade will be based on (a)
attendance, (b) level of engagement during lectures and labs, (c) level
of engagement on Slack, and (d) the completion of other exercises
including ``entry'' and ``exit'' tickets, the student information sheet,
a pre-test, and an end of the semester course evaluation.

Each of these elements is assigned a point value and assessed using a
scale that awards full, partial, or no credit. Your participation grade
will be split, with 50 points (5\% of your final grade) for the first
half of the semester (through Lecture-07) and another 50 points (5\%)
for the second half. Since the number of points awarded for
participation are variable, the total number of points earned for each
half will be converted to a 0 to 50 scale.

I provide the final number of points earned for each half of the course.
If you would like a more detailed breakdown of your participation grade,
please reach out and I will provide one.

\subsection{Lecture Preps}\label{lecture-preps}

\begin{rmdtip}
Lecture preps are worth \textbf{6\%} of your final grade.
\end{rmdtip}

Before each course meeting, you will need to complete all assigned
readings. For a part of these readings, you will also need to complete a
textbook exercise. These prep exercises are designed to get you ready
for the week's material by exposing you to basic, guided examples before
class begins. Instructions for the lecture preps will be posted in the
lecture repositories on
\href{https://github.com/slu-soc5050}{\textbf{GitHub}} and will be
linked to from the lecture pages on the
\href{https://slu-soc5050.github.io/}{\textbf{course website}}. The
instructions will also detail the deliverables to be submitted to
demonstrate completion of each assignment.

For many of the lecture preps, I will post a YouTube video of me
completing the exercise and narrating the process. These videos will be
embedded in the lecture pages on the
\href{https://slu-soc5050.github.io/}{\textbf{course website}}. You
should follow along with the video and use it as a guide for completing
the exercise yourself. I will also post replication files that detail
the process and, if relevant, the code for completing the lecture prep.
Like the instructions, these will be posted in the lecture repositories
on \href{https://github.com/slu-soc5050}{\textbf{GitHub}}.

There will be a total of fifteen lecture preps over the course of the
semester, each of which is worth 4 points (0.4\% of your final grade).
Lecture preps are graded using the ``check'' grading system. Since
replication files are posted, feedback for lecture preps is not
generally returned and we will only respond with the number of points
awarded if you do not earn full credit.

\subsection{Lab Exercises}\label{lab-exercises}

\begin{rmdtip}
Labs are worth \textbf{15\%} of your final grade.
\end{rmdtip}

Each course meeting (except the first) will include time dedicated to
practicing the techniques and applying the theories described during the
day's lecture. These exercises will give you an opportunity to practice
skills that correspond with the first four course objectives.
Instructions for the labs will be posted in the lecture repositories on
\href{https://github.com/slu-soc5050}{\textbf{GitHub}} and will be
linked to from the lecture pages on the
\href{https://slu-soc5050.github.io/}{\textbf{course website}}. The
instructions will also detail the deliverables to be submitted to
demonstrate completion of each assignment. Replication files are also
provided in the lecture repositories on
\href{https://github.com/slu-soc5050}{\textbf{GitHub}}.

Lab exercises will be completed in small workgroups, though each student
is expected to turn in the required deliverables. We will assign
students to workgroups and may shuffle their composition over the course
of the semester. Completing a lab entails not just successfully
submitting the required deliverables but also actively contributing to
the group discussions that help to produce them.

There will be a total of fifteen lab exercises over the course of the
semester, each of which is worth 10 points (1.5\% of your final grade).
Lab exercises are graded using the ``check'' grading system. Since
replication files are posted, feedback for labs is not generally
returned and we will only respond with the number of points awarded if
you do not earn full credit.

\subsection{Problem Sets}\label{problem-sets}

\begin{rmdtip}
Problem sets are worth \textbf{28\%} of your final grade.
\end{rmdtip}

Problem sets will require students to draw on a variety of skills,
including cleaning data, performing statistical analyses, producing
plots, and reporting results. They are designed to assess your progress
with the first four course objectives. Instructions for the problem sets
will be posted in the lecture repositories on
\href{https://github.com/slu-soc5050}{\textbf{GitHub}} and will be
linked to from the lecture pages on the
\href{https://slu-soc5050.github.io/}{\textbf{course website}}. The
instructions will also detail the deliverables to be submitted to
demonstrate completion of each assignment. Replication files that
illustrate my approach to each problem set will be posted on
\href{https://github.com/slu-soc5050}{\textbf{GitHub}} in the
\href{https://github.com/slu-soc5050/Replications}{\texttt{Replications}}
repository once all students have submitted their problem sets.

There will be a total of eight problem sets over the course of the
semester, each of which is worth 35 points (3.5\% of your final grade).
Each Problem Set will include a simple rubric describing how each
problem set is evaluated. A key aspect of these assignments is not only
demonstrating comfort with a particular set of GISc skills, but also
demonstrating and evolving in your analysis development, programming,
and cartography skills as well. The weight given to quality of your
process, code, and graphic design will increase as the semester
progresses.

\subsection{Final Project}\label{final-project}

\begin{rmdtip}
The final project is worth, in total, \textbf{41\%} of your final grade.
Depending on your section, it will be broken down into a variety of
assignments, each of which has their own point value. See below for
details.
\end{rmdtip}

The final project corresponds with the fourth learning outcome. It will
be organized slightly differently depending on which section you are
enrolled in. Specific instructions will be provided in the
\href{https://slu-soc5050.github.io/finalGuide}{\textbf{final project
guide}}, and updates will be posted on the
\href{https://slu-soc5050.github.io/}{\textbf{course website's}}
\href{https://slu-soc5050.github.io/final-project}{\textbf{final project
page}}.

In brief, all students will select a topic and submit their topic by
Lecture-03 (\textbf{September 10\textsuperscript{th}}) as an ``Issue''
in their individual
\href{https://github.com/slu-soc5650}{\textbf{GitHub}} assignments
repository. Groups will be formed based on topic area. These groups will
be used for support throughout the semester as well as peer review of
particular pieces of the project itself.

As work progresses, there will be a number of \textbf{waypoints} where
students will need to submit updates on their progress. Waypoints beyond
the memo submission are as follows:

\begin{itemize}
\tightlist
\item
  Lecture-05 (\textbf{September 24\textsuperscript{th}}) - Meeting
  report posted in each group's Slack channel
\item
  Lecture-08 (\textbf{October 15\textsuperscript{th}}) - Progress report
  from each student due in group's Slack channel
\item
  Lecture-11 (\textbf{November 5\textsuperscript{th}}) - Draft materials
  due in each student's final project repository
\item
  Lecture-12 (\textbf{November 12\textsuperscript{th}}) - Peer reviews
  due to group members as a GitHub issue in each student's final project
  repository
\item
  Lecture-13 (\textbf{November 19\textsuperscript{th}}) - Response to
  reviewer due in the GitHub issue opened by the reviewer
\item
  Lecture-15 (\textbf{December 3\textsuperscript{rd}}) - Progress report
  from each student due as a GitHub issue in each student's final
  project repository
\end{itemize}

Deliverables for each waypoint are described in the
\href{https://slu-soc5050.github.io/finalGuide}{\textbf{final project
guide}}. All waypoints are graded using the ``check'' grading system.
Final materials will be due on \textbf{December 17\textsuperscript{th}}
(during Finals Week), when we will hold a ``research conference'' in
Morrissey Hall. During our conference, each student will present their
results using PowerPoint (or similar). Final deliverables differ by
course section.

\subsubsection{SOC 4015}\label{soc-4015}

If you are enrolled in SOC 4015, you will need to pick a continuous
variable from the 2012 General Social Survey to use as your main study
variable. You will then clean the data and conduct an analysis of this
variable using a variety of statistical tests covered this semester.
Your final results will be presented as a PowerPoint presentation during
finals week.

\begin{table}

\caption{\label{tab:unnamed-chunk-6}SOC 4015 Final Project Breakdown}
\centering
\begin{tabular}[t]{llll}
\toprule
Assignment & Points & Quantity & Total\\
\midrule
Memo & 20 pts & x1 & 20 pts\\
Waypoints & 20 pts & x6 & 120 pts\\
Draft Code \& Docs & 20 pts & x1 & 20 pts\\
Draft Slides & 20 pts & x1 & 20 pts\\
Final Code \& Docs & 100 pts & x1 & 100 pts\\
\addlinespace
Final Slides & 100 pts & x1 & 100 pts\\
Final Presentation & 30 pts & x1 & 30 pts\\
\bottomrule
\end{tabular}
\end{table}

\paragraph{SOC 5050}\label{soc-5050}

If you are enrolled in SOC 5050, you will need to identify an
appropriate data set that contains a continuous variable that you can
use as your main study variable. You will then clean the data and
conduct an analysis of this variable using a variety of statistical
tests covered this semester. Your final results will be presented as a
PowerPoint presentation during finals week.

You will also have to produce a 5,000 word final journal article
manuscript that places your project in the relevant social science
literature, presents your data and methods, and provides a summary and
discussion of your results. An annotated bibliography will be due at
Lecture-07 (\textbf{October 8\textsuperscript{th}}) and the draft paper
will be due at Lecture-12 (\textbf{Novemeber 12\textsuperscript{th}};
note that this is one week \emph{after} the other draft materials).

\begin{table}

\caption{\label{tab:unnamed-chunk-7}SOC 5050 Final Project Breakdown}
\centering
\begin{tabular}[t]{llll}
\toprule
Assignment & Points & Quantity & Total\\
\midrule
Memo & 10 pts & x1 & 10 pts\\
Waypoints & 10 pts & x6 & 60 pts\\
Annotated Bibliography & 15 pts & x1 & 15 pts\\
Draft Code \& Docs & 15 pts & x1 & 15 pts\\
Draft Slides & 15 pts & x1 & 15 pts\\
\addlinespace
Draft Paper & 15 pts & x1 & 15 pts\\
Final Code \& Docs & 50 pts & x1 & 50 pts\\
Final Slides & 100 pts & x1 & 100 pts\\
Final Presentation & 30 pts & x1 & 30 pts\\
Final Paper & 100 pts & x1 & 100 pts\\
\bottomrule
\end{tabular}
\end{table}

\section{Submission and Late Work}\label{submission-and-late-work}

\subsection{Assignment Submission}\label{assignment-submission}

Copies of all assignment requested deliverables should be uploaded to
your private assignments repository on
\href{https://github.com/slu-soc5650}{GitHub} before class on the day
that the assignments are due. All assignments will contain details on
required deliverables.

The GitHub submission policy is in place because it facilitates clear,
easy grading that can be turned around to you quickly. Submitting
assignments in ways that deviate from this policy will result in a late
grade (see below) being applied in the first instance and a zero grade
for each subsequent instance.

\subsection{Licensing of Student Work}\label{licensing-of-student-work}

All assignment repositories are licensed under a
\href{https://creativecommons.org/licenses/by-nc-nd/4.0/}{Creative
Commons Attribution-NonCommercial-NoDerivatives 4.0 International
License}. This license explicitly gives you copyright to all work you
create for this course. The license gives Chris permission to copy your
work (such as for grading) and to re-use your work later for
non-commercial purposes (such as in-class examples) so long as you are
given credit for it. However, your work cannot be used for monetary gain
(such as in a textbook) and derivative works based on your work are
prohibited.

The syllabus agreement at the end of the Student Information Sheet
includes an acknowledgement of this licensing arrangement. If you have
questions about this, please contact Chris before submitting the form.

\subsection{Late Work}\label{late-work}

Once the class begins, any assignments submitted will be treated as
late. Assignments handed in within 24-hours of the beginning of class
will have 15\% deducted from the grade. I will deduct 15\% per day for
the next two 24-hour periods that assignments are late. After 72 hours,
I will not accept late work. If you cannot attend class because of
personal illness, a family issue, jury duty, an athletic match, or a
religious observance, you must contact me beforehand to discuss
alternate submission of work.

\section{Extra Credit}\label{extra-credit}

From time to time I may offer extra credit to be applied to your final
grade. I will only offer extra credit if it is open to the entire class
(typically for something like attending a lecture or event on-campus).
If I offer extra credit, I will typically require you to submit a short
written summary of the activity within a week of the event to obtain the
credit. When offered, extra credit opportunities cannot be made-up or
substituted if you are unable to attend the event.

\section{Grading}\label{grading}

Grades will be included with assignment feedback, which will be
disseminated through Github's \textbf{Issues} tool. At midterms, Lecture
15, and finals, I will upload a summary of all assignment grades to a
new \textbf{Issue} on GitHub.

All grades that use a ``check'' system (the lecture preps, labs, and
some aspects of the final project) will be calculated using the
following approach. A ``check-plus'' represents excellent work and will
get full credit. A ``check'' represents satisfactory work and will get
85\% of the points available for that assignment. A ``check-minus''
represents work that needs substantial improvement and will get 75\% of
the points available for that assignment.

I use a point system for calculating grades. The following table gives
the weighting and final point totals for all assignments for this
course:

\begin{table}

\caption{\label{tab:unnamed-chunk-8}SOC 4015 & 5050 Points Breakdown}
\centering
\begin{tabular}[t]{lllll}
\toprule
Assignment & Points & Quantity & Total & Percent\\
\midrule
Participation & 50 pts & x2 & 100 pts & 10\%\\
Lecture Preps & 4 pts & x15 & 60 pts & 6\%\\
Labs & 10 pts & x5 & 150 pts & 15\%\\
Problem Sets & 35 pts & x8 & 280 pts & 28\%\\
Final Project & 410 pts & x1 & 410 pts & 41\%\\
\bottomrule
\end{tabular}
\end{table}

All feedback will include grades that represent number of points earned.
If you want to know your percentage on a particular assignment, divide
the number of points earned by the number of points possible and then
multiply it by 100.

Final grades will be calculated by taking the sum of all points earned
and dividing it by the total number of points possible (1,000). This
will be multiplied by 100 and then converted to a letter grade using the
following table:

\begin{table}
\caption{\label{tab:unnamed-chunk-9}Course Grading Scale}

\centering
\begin{tabular}[t]{lll}
\toprule
GPA & Letter & Percent\\
\midrule
4.0 & A & 93.0\% - 100\%\\
3.7 & A- & 90.0\% - 92.9\%\\
3.3 & B+ & 87.0\% - 89.9\%\\
3.0 & B & 83.0\% - 86.9\%\\
2.7 & B- & 80.0\% - 82.9\%\\
\bottomrule
\end{tabular}
\centering
\begin{tabular}[t]{lll}
\toprule
GPA & Letter & Percent\\
\midrule
2.3 & C+ & 77.0\% - 79.9\%\\
2.0 & C & 73.0\% - 76.9\%\\
1.7 & C- & 70.0\% - 72.9\%\\
1.0 & D & 63.0\% - 69.9\%\\
0.0 & F & < 63.0\%\\
\bottomrule
\end{tabular}
\end{table}

No chances will be given for revisions of poor grades. Incomplete grades
will be given upon request only if you have a ``C'' average and have
completed at least two-thirds of the assignments. You should note that
incomplete grades must be rectified by the specified deadline or they
convert to an ``F''.

\begin{rmdwarning}
No chances will be given for revisions of poor grades. Incomplete grades
will be given upon request only if you have a ``C'' average and have
completed at least two-thirds of the assignments. You should note that
incomplete grades must be rectified by the specified deadline or they
convert to an ``F''.
\end{rmdwarning}

\part{Reading List}\label{part-reading-list}

\chapter{Course Schedule}\label{course-schedule}

The following is a high-level schedule that details the general topic
covered by each lecture.

\begin{table}

\caption{\label{tab:unnamed-chunk-1}SOC 4015 & 5050 Course Overview}
\centering
\begin{tabular}[t]{lll}
\toprule
Lecture & Date & Topic\\
\midrule
 & prior to August 27\textasciicircum{}th\textasciicircum{} & Course Preview\\
01 & August 27\textasciicircum{}th\textasciicircum{} & Course Introduction\\
02 & September 3\textasciicircum{}rd\textasciicircum{} & Working with Data\\
03 & September 10\textasciicircum{}th\textasciicircum{} & Describing Distributions\\
04 & September 17\textasciicircum{}th\textasciicircum{} & Probability\\
\addlinespace
05 & September 24\textasciicircum{}th\textasciicircum{} & The Distribution of Random Variables\\
06 & October 1\textasciicircum{}st\textasciicircum{} & Foundations for Inference\\
07 & October 8\textasciicircum{}th\textasciicircum{} & Difference of Means (Part 1)\\
08 & October 15\textasciicircum{}th\textasciicircum{} & Difference of Means (Part 2)\\
09 & October 22\textasciicircum{}nd\textasciicircum{} & Working with Factors\\
\addlinespace
10 & October 29\textasciicircum{}th\textasciicircum{} & Correlations (Part 1)\\
11 & November 5\textasciicircum{}th\textasciicircum{} & Correlations (Part 2)\\
12 & November 12\textasciicircum{}th\textasciicircum{} & OLS Regression (Part 1)\\
13 & November 19\textasciicircum{}th\textasciicircum{} & OLS Regression (Part 2)\\
14 & November 26\textasciicircum{}th\textasciicircum{} & OLS Regression (Part 3)\\
\addlinespace
15 & December 3\textasciicircum{}rd\textasciicircum{} & Analysis of Variance\\
16 & December 10\textasciicircum{}th\textasciicircum{} & Chi-Squared\\
 & December 17\textasciicircum{}th\textasciicircum{} & Final Presentations\\
\bottomrule
\end{tabular}
\end{table}

\section{Planned Online Lectures}\label{planned-online-lectures}

This semester, we have two classes that fall on official university
holidays: Labor Day (Lecture-02, \textbf{September
3\textsuperscript{rd}}) and Fall Break (Lecture-09, \textbf{October
22\textsuperscript{nd}}). These weeks will have materials assigned for
them, which will include lectures posted on YouTube. These lectures will
be shorter than typical in-class lectures. Students should view these
lectures during that week and complete the associated readings and lab
exercises. Videos will be embedded in the lecture pages on the
\href{https://slu-soc5050.github.io/}{\textbf{course website}}.

\section{Class Progression}\label{class-progression}

Each lecture will be broken down roughly the same way. Students are
expected to arrive having already completed the previous week's work as
well as the assigned readings and lecture prep. Class will begin with
any relevant ``front matter'' including follow-up from the previous
weeks and relevant announcements. When assigned, we will then segue into
a discussion of the Wheelan text before spending the majority of class
focused on the day's main topic, which will typically be related to one
or more of the first three learning outcomes. Around 6:00pm, we will
take a short break. Most classes will end with time dedicated to working
through the lab exercise. After class, students are expected to finish
the lab and, if necessary, the assigned problem set as well.

\section{Scheduling Notes}\label{scheduling-notes}

The lecture schedule may change as it depends on the progress of the
class. However, you must keep up with the reading assignments. In the
event of a cancellation due to weather or another disruption, I may
alter the lecture schedule.

Since this course only meets once per week, cancellations are
particularly disruptive. I will make every effort to schedule make-up
classes at a time that works for at least a portion of the class. These
class sessions will be recorded and made immediately available using
YouTube for students who are unable to attend the make-up class. All
students will be responsible for either attending the make-up class or
watching the lecture as well as completing all readings, lab
assignments, and problem sets for make-up classes.

\chapter{Abbreviations}\label{abbreviations}

The following abbreviations are used in this document.

\chapter{Weekly Schedule}\label{weekly-schedule}

Select a lecture from the menu to see details about topics, readings,
and assignments. Additional notes and links to course materials are
available through the \href{https://slu-soc5050.github.io}{course
website}, which has dedicated pages for each lecture. Links to these
pages are included on each lecture's reading list entry.

\section{Course Preview}\label{course-preview}

View on Course Website

\subsection*{Topics}\label{topics}
\addcontentsline{toc}{subsection}{Topics}

\begin{itemize}
\tightlist
\item
  \textbf{Inferential Statistics:}
\item
  \textbf{Data Visualization:}
\item
  \textbf{Data Analysis:}
\item
  \textbf{Quantitative Research:}
\end{itemize}

\subsection*{Readings}\label{readings-1}
\addcontentsline{toc}{subsection}{Readings}

\subsubsection*{Required}\label{required}
\addcontentsline{toc}{subsubsection}{Required}

\begin{itemize}
\item
\item
\end{itemize}

\subsubsection*{\texorpdfstring{\emph{Optional}}{Optional}}\label{optional}
\addcontentsline{toc}{subsubsection}{\emph{Optional}}

\begin{itemize}
\item
\end{itemize}

\subsection*{Assignments}\label{assignments-1}
\addcontentsline{toc}{subsection}{Assignments}

\subsubsection*{\texorpdfstring{Due Before \emph{Next}
Class}{Due Before Next Class}}\label{due-before-next-class}
\addcontentsline{toc}{subsubsection}{Due Before \emph{Next} Class}

\begin{itemize}
\tightlist
\item
  Lecture Prep 01 - Course Onboarding
\item
  Lecture Prep 02 - Course Preview
\end{itemize}

\section{Lecture-01: Course
Introduction}\label{lecture-01-course-introduction}

View on Course Website

\subsection*{Topics}\label{topics-1}
\addcontentsline{toc}{subsection}{Topics}

\begin{itemize}
\tightlist
\item
  \textbf{Syllabus Overview}
\item
  \textbf{Inferential Statistics:} Defining quantitative data
\item
  \textbf{Data Analysis:} Intro to \texttt{R} and RStudio
\item
  \textbf{Quantitative Research:} What is a workflow?
\end{itemize}

\subsection*{Readings}\label{readings-2}
\addcontentsline{toc}{subsection}{Readings}

\subsubsection*{Required}\label{required-1}
\addcontentsline{toc}{subsubsection}{Required}

\begin{itemize}
\tightlist
\item
  OpenIntro: Chapter 1, pages 7-26 (ER)
\item
  R4DS:

  \begin{itemize}
  \tightlist
  \item
    \emph{Print} - Preface \textbf{\emph{or}}
  \item
    \emph{Web} - Chapter 1 (Link)
  \end{itemize}
\item
  SSDS - Chapters 1 through 5 (Link)
\item
  Wheelan - Chapter 1 (ER)
\end{itemize}

\subsubsection*{\texorpdfstring{\emph{Optional}}{Optional}}\label{optional-1}
\addcontentsline{toc}{subsubsection}{\emph{Optional}}

\begin{itemize}
\tightlist
\item
  Wilson, G., Bryan, J., Cranston, K., Kitzes, J., Nederbragt, L., \&
  Teal, T. K. (2017). Good enough practices in scientific computing.
  \emph{PLoS computational biology}, 13(6), e1005510. (ER)
\end{itemize}

\subsection*{Assignments}\label{assignments-2}
\addcontentsline{toc}{subsection}{Assignments}

\subsubsection*{Due Before Class}\label{due-before-class}
\addcontentsline{toc}{subsubsection}{Due Before Class}

\begin{itemize}
\tightlist
\item
  Lecture Prep 01 - Course Onboarding
\item
  Lecture Prep 02 - Course Preview
\end{itemize}

\subsubsection*{\texorpdfstring{Due Before \emph{Next}
Class}{Due Before Next Class}}\label{due-before-next-class-1}
\addcontentsline{toc}{subsubsection}{Due Before \emph{Next} Class}

\begin{itemize}
\tightlist
\item
  \emph{none}
\end{itemize}

\section{Lecture-02: Working with
Data}\label{lecture-02-working-with-data}

View on Course Website

\subsection*{Topics}\label{topics-2}
\addcontentsline{toc}{subsection}{Topics}

\begin{itemize}
\tightlist
\item
  \textbf{Inferential Statistics:}
\item
  \textbf{Data Visualization:}
\item
  \textbf{Data Analysis:}
\item
  \textbf{Quantitative Research:}
\end{itemize}

\subsection*{Readings}\label{readings-3}
\addcontentsline{toc}{subsection}{Readings}

\subsubsection*{Required}\label{required-2}
\addcontentsline{toc}{subsubsection}{Required}

\begin{itemize}
\item
\item
\end{itemize}

\subsubsection*{\texorpdfstring{\emph{Optional}}{Optional}}\label{optional-2}
\addcontentsline{toc}{subsubsection}{\emph{Optional}}

\begin{itemize}
\item
\end{itemize}

\subsection*{Assignments}\label{assignments-3}
\addcontentsline{toc}{subsection}{Assignments}

\subsubsection*{Due Before Class}\label{due-before-class-1}
\addcontentsline{toc}{subsubsection}{Due Before Class}

\begin{itemize}
\item
\end{itemize}

\subsubsection*{\texorpdfstring{Due Before \emph{Next}
Class}{Due Before Next Class}}\label{due-before-next-class-2}
\addcontentsline{toc}{subsubsection}{Due Before \emph{Next} Class}

\begin{itemize}
\item
\end{itemize}

\section{Lecture-03: Describing
Distributions}\label{lecture-03-describing-distributions}

View on Course Website

\subsection*{Topics}\label{topics-3}
\addcontentsline{toc}{subsection}{Topics}

\begin{itemize}
\tightlist
\item
  \textbf{Inferential Statistics:}
\item
  \textbf{Data Visualization:}
\item
  \textbf{Data Analysis:}
\item
  \textbf{Quantitative Research:}
\end{itemize}

\subsection*{Readings}\label{readings-4}
\addcontentsline{toc}{subsection}{Readings}

\subsubsection*{Required}\label{required-3}
\addcontentsline{toc}{subsubsection}{Required}

\begin{itemize}
\item
\item
\end{itemize}

\subsubsection*{\texorpdfstring{\emph{Optional}}{Optional}}\label{optional-3}
\addcontentsline{toc}{subsubsection}{\emph{Optional}}

\begin{itemize}
\item
\end{itemize}

\subsection*{Assignments}\label{assignments-4}
\addcontentsline{toc}{subsection}{Assignments}

\subsubsection*{Due Before Class}\label{due-before-class-2}
\addcontentsline{toc}{subsubsection}{Due Before Class}

\begin{itemize}
\item
\end{itemize}

\subsubsection*{\texorpdfstring{Due Before \emph{Next}
Class}{Due Before Next Class}}\label{due-before-next-class-3}
\addcontentsline{toc}{subsubsection}{Due Before \emph{Next} Class}

\begin{itemize}
\item
\end{itemize}

\section{Lecture-04: Probability}\label{lecture-04-probability}

View on Course Website

\subsection*{Topics}\label{topics-4}
\addcontentsline{toc}{subsection}{Topics}

\begin{itemize}
\tightlist
\item
  \textbf{Inferential Statistics:}
\item
  \textbf{Data Visualization:}
\item
  \textbf{Data Analysis:}
\item
  \textbf{Quantitative Research:}
\end{itemize}

\subsection*{Readings}\label{readings-5}
\addcontentsline{toc}{subsection}{Readings}

\subsubsection*{Required}\label{required-4}
\addcontentsline{toc}{subsubsection}{Required}

\begin{itemize}
\item
\item
\end{itemize}

\subsubsection*{\texorpdfstring{\emph{Optional}}{Optional}}\label{optional-4}
\addcontentsline{toc}{subsubsection}{\emph{Optional}}

\begin{itemize}
\item
\end{itemize}

\subsection*{Assignments}\label{assignments-5}
\addcontentsline{toc}{subsection}{Assignments}

\subsubsection*{Due Before Class}\label{due-before-class-3}
\addcontentsline{toc}{subsubsection}{Due Before Class}

\begin{itemize}
\item
\end{itemize}

\subsubsection*{\texorpdfstring{Due Before \emph{Next}
Class}{Due Before Next Class}}\label{due-before-next-class-4}
\addcontentsline{toc}{subsubsection}{Due Before \emph{Next} Class}

\begin{itemize}
\item
\end{itemize}

\section{Lecture-05: The Distribution of Random
Variables}\label{lecture-05-the-distribution-of-random-variables}

View on Course Website

\subsection*{Topics}\label{topics-5}
\addcontentsline{toc}{subsection}{Topics}

\begin{itemize}
\tightlist
\item
  \textbf{Inferential Statistics:}
\item
  \textbf{Data Visualization:}
\item
  \textbf{Data Analysis:}
\item
  \textbf{Quantitative Research:}
\end{itemize}

\subsection*{Readings}\label{readings-6}
\addcontentsline{toc}{subsection}{Readings}

\subsubsection*{Required}\label{required-5}
\addcontentsline{toc}{subsubsection}{Required}

\begin{itemize}
\item
\item
\end{itemize}

\subsubsection*{\texorpdfstring{\emph{Optional}}{Optional}}\label{optional-5}
\addcontentsline{toc}{subsubsection}{\emph{Optional}}

\begin{itemize}
\item
\end{itemize}

\subsection*{Assignments}\label{assignments-6}
\addcontentsline{toc}{subsection}{Assignments}

\subsubsection*{Due Before Class}\label{due-before-class-4}
\addcontentsline{toc}{subsubsection}{Due Before Class}

\begin{itemize}
\item
\end{itemize}

\subsubsection*{\texorpdfstring{Due Before \emph{Next}
Class}{Due Before Next Class}}\label{due-before-next-class-5}
\addcontentsline{toc}{subsubsection}{Due Before \emph{Next} Class}

\begin{itemize}
\item
\end{itemize}

\section{Lecture-06: Foundations for
Inference}\label{lecture-06-foundations-for-inference}

View on Course Website

\subsection*{Topics}\label{topics-6}
\addcontentsline{toc}{subsection}{Topics}

\begin{itemize}
\tightlist
\item
  \textbf{Inferential Statistics:}
\item
  \textbf{Data Visualization:}
\item
  \textbf{Data Analysis:}
\item
  \textbf{Quantitative Research:}
\end{itemize}

\subsection*{Readings}\label{readings-7}
\addcontentsline{toc}{subsection}{Readings}

\subsubsection*{Required}\label{required-6}
\addcontentsline{toc}{subsubsection}{Required}

\begin{itemize}
\item
\item
\end{itemize}

\subsubsection*{\texorpdfstring{\emph{Optional}}{Optional}}\label{optional-6}
\addcontentsline{toc}{subsubsection}{\emph{Optional}}

\begin{itemize}
\item
\end{itemize}

\subsection*{Assignments}\label{assignments-7}
\addcontentsline{toc}{subsection}{Assignments}

\subsubsection*{Due Before Class}\label{due-before-class-5}
\addcontentsline{toc}{subsubsection}{Due Before Class}

\begin{itemize}
\item
\end{itemize}

\subsubsection*{\texorpdfstring{Due Before \emph{Next}
Class}{Due Before Next Class}}\label{due-before-next-class-6}
\addcontentsline{toc}{subsubsection}{Due Before \emph{Next} Class}

\begin{itemize}
\item
\end{itemize}

\section{Lecture-07: Difference of Means (Part
1)}\label{lecture-07-difference-of-means-part-1}

View on Course Website

\subsection*{Topics}\label{topics-7}
\addcontentsline{toc}{subsection}{Topics}

\begin{itemize}
\tightlist
\item
  \textbf{Inferential Statistics:}
\item
  \textbf{Data Visualization:}
\item
  \textbf{Data Analysis:}
\item
  \textbf{Quantitative Research:}
\end{itemize}

\subsection*{Readings}\label{readings-8}
\addcontentsline{toc}{subsection}{Readings}

\subsubsection*{Required}\label{required-7}
\addcontentsline{toc}{subsubsection}{Required}

\begin{itemize}
\item
\item
\end{itemize}

\subsubsection*{\texorpdfstring{\emph{Optional}}{Optional}}\label{optional-7}
\addcontentsline{toc}{subsubsection}{\emph{Optional}}

\begin{itemize}
\item
\end{itemize}

\subsection*{Assignments}\label{assignments-8}
\addcontentsline{toc}{subsection}{Assignments}

\subsubsection*{Due Before Class}\label{due-before-class-6}
\addcontentsline{toc}{subsubsection}{Due Before Class}

\begin{itemize}
\item
\end{itemize}

\subsubsection*{\texorpdfstring{Due Before \emph{Next}
Class}{Due Before Next Class}}\label{due-before-next-class-7}
\addcontentsline{toc}{subsubsection}{Due Before \emph{Next} Class}

\begin{itemize}
\item
\end{itemize}

\section{Lecture-08: Difference of Means (Part
2)}\label{lecture-08-difference-of-means-part-2}

View on Course Website

\subsection*{Topics}\label{topics-8}
\addcontentsline{toc}{subsection}{Topics}

\begin{itemize}
\tightlist
\item
  \textbf{Inferential Statistics:}
\item
  \textbf{Data Visualization:}
\item
  \textbf{Data Analysis:}
\item
  \textbf{Quantitative Research:}
\end{itemize}

\subsection*{Readings}\label{readings-9}
\addcontentsline{toc}{subsection}{Readings}

\subsubsection*{Required}\label{required-8}
\addcontentsline{toc}{subsubsection}{Required}

\begin{itemize}
\item
\item
\end{itemize}

\subsubsection*{\texorpdfstring{\emph{Optional}}{Optional}}\label{optional-8}
\addcontentsline{toc}{subsubsection}{\emph{Optional}}

\begin{itemize}
\item
\end{itemize}

\subsection*{Assignments}\label{assignments-9}
\addcontentsline{toc}{subsection}{Assignments}

\subsubsection*{Due Before Class}\label{due-before-class-7}
\addcontentsline{toc}{subsubsection}{Due Before Class}

\begin{itemize}
\item
\end{itemize}

\subsubsection*{\texorpdfstring{Due Before \emph{Next}
Class}{Due Before Next Class}}\label{due-before-next-class-8}
\addcontentsline{toc}{subsubsection}{Due Before \emph{Next} Class}

\begin{itemize}
\item
\end{itemize}

\section{Lecture-09: Working with
Factors}\label{lecture-09-working-with-factors}

View on Course Website

\subsection*{Topics}\label{topics-9}
\addcontentsline{toc}{subsection}{Topics}

\begin{itemize}
\tightlist
\item
  \textbf{Inferential Statistics:}
\item
  \textbf{Data Visualization:}
\item
  \textbf{Data Analysis:}
\item
  \textbf{Quantitative Research:}
\end{itemize}

\subsection*{Readings}\label{readings-10}
\addcontentsline{toc}{subsection}{Readings}

\subsubsection*{Required}\label{required-9}
\addcontentsline{toc}{subsubsection}{Required}

\begin{itemize}
\item
\item
\end{itemize}

\subsubsection*{\texorpdfstring{\emph{Optional}}{Optional}}\label{optional-9}
\addcontentsline{toc}{subsubsection}{\emph{Optional}}

\begin{itemize}
\item
\end{itemize}

\subsection*{Assignments}\label{assignments-10}
\addcontentsline{toc}{subsection}{Assignments}

\subsubsection*{Due Before Class}\label{due-before-class-8}
\addcontentsline{toc}{subsubsection}{Due Before Class}

\begin{itemize}
\item
\end{itemize}

\subsubsection*{\texorpdfstring{Due Before \emph{Next}
Class}{Due Before Next Class}}\label{due-before-next-class-9}
\addcontentsline{toc}{subsubsection}{Due Before \emph{Next} Class}

\begin{itemize}
\item
\end{itemize}

\section{Lecture-10: Correlations (Part
1)}\label{lecture-10-correlations-part-1}

View on Course Website

\subsection*{Topics}\label{topics-10}
\addcontentsline{toc}{subsection}{Topics}

\begin{itemize}
\tightlist
\item
  \textbf{Inferential Statistics:}
\item
  \textbf{Data Visualization:}
\item
  \textbf{Data Analysis:}
\item
  \textbf{Quantitative Research:}
\end{itemize}

\subsection*{Readings}\label{readings-11}
\addcontentsline{toc}{subsection}{Readings}

\subsubsection*{Required}\label{required-10}
\addcontentsline{toc}{subsubsection}{Required}

\begin{itemize}
\item
\item
\end{itemize}

\subsubsection*{\texorpdfstring{\emph{Optional}}{Optional}}\label{optional-10}
\addcontentsline{toc}{subsubsection}{\emph{Optional}}

\begin{itemize}
\item
\end{itemize}

\subsection*{Assignments}\label{assignments-11}
\addcontentsline{toc}{subsection}{Assignments}

\subsubsection*{Due Before Class}\label{due-before-class-9}
\addcontentsline{toc}{subsubsection}{Due Before Class}

\begin{itemize}
\item
\end{itemize}

\subsubsection*{\texorpdfstring{Due Before \emph{Next}
Class}{Due Before Next Class}}\label{due-before-next-class-10}
\addcontentsline{toc}{subsubsection}{Due Before \emph{Next} Class}

\begin{itemize}
\item
\end{itemize}

\section{Lecture-11: Correlations (Part
2)}\label{lecture-11-correlations-part-2}

View on Course Website

\subsection*{Topics}\label{topics-11}
\addcontentsline{toc}{subsection}{Topics}

\begin{itemize}
\tightlist
\item
  \textbf{Inferential Statistics:}
\item
  \textbf{Data Visualization:}
\item
  \textbf{Data Analysis:}
\item
  \textbf{Quantitative Research:}
\end{itemize}

\subsection*{Readings}\label{readings-12}
\addcontentsline{toc}{subsection}{Readings}

\subsubsection*{Required}\label{required-11}
\addcontentsline{toc}{subsubsection}{Required}

\begin{itemize}
\item
\item
\end{itemize}

\subsubsection*{\texorpdfstring{\emph{Optional}}{Optional}}\label{optional-11}
\addcontentsline{toc}{subsubsection}{\emph{Optional}}

\begin{itemize}
\item
\end{itemize}

\subsection*{Assignments}\label{assignments-12}
\addcontentsline{toc}{subsection}{Assignments}

\subsubsection*{Due Before Class}\label{due-before-class-10}
\addcontentsline{toc}{subsubsection}{Due Before Class}

\begin{itemize}
\item
\end{itemize}

\subsubsection*{\texorpdfstring{Due Before \emph{Next}
Class}{Due Before Next Class}}\label{due-before-next-class-11}
\addcontentsline{toc}{subsubsection}{Due Before \emph{Next} Class}

\begin{itemize}
\item
\end{itemize}

\section{Lecture-12: OLS Regression (Part
1)}\label{lecture-12-ols-regression-part-1}

View on Course Website

\subsection*{Topics}\label{topics-12}
\addcontentsline{toc}{subsection}{Topics}

\begin{itemize}
\tightlist
\item
  \textbf{Inferential Statistics:}
\item
  \textbf{Data Visualization:}
\item
  \textbf{Data Analysis:}
\item
  \textbf{Quantitative Research:}
\end{itemize}

\subsection*{Readings}\label{readings-13}
\addcontentsline{toc}{subsection}{Readings}

\subsubsection*{Required}\label{required-12}
\addcontentsline{toc}{subsubsection}{Required}

\begin{itemize}
\item
\item
\end{itemize}

\subsubsection*{\texorpdfstring{\emph{Optional}}{Optional}}\label{optional-12}
\addcontentsline{toc}{subsubsection}{\emph{Optional}}

\begin{itemize}
\item
\end{itemize}

\subsection*{Assignments}\label{assignments-13}
\addcontentsline{toc}{subsection}{Assignments}

\subsubsection*{Due Before Class}\label{due-before-class-11}
\addcontentsline{toc}{subsubsection}{Due Before Class}

\begin{itemize}
\item
\end{itemize}

\subsubsection*{\texorpdfstring{Due Before \emph{Next}
Class}{Due Before Next Class}}\label{due-before-next-class-12}
\addcontentsline{toc}{subsubsection}{Due Before \emph{Next} Class}

\begin{itemize}
\item
\end{itemize}

\section{Lecture-13: OLS Regression (Part
2)}\label{lecture-13-ols-regression-part-2}

View on Course Website

\subsection*{Topics}\label{topics-13}
\addcontentsline{toc}{subsection}{Topics}

\begin{itemize}
\tightlist
\item
  \textbf{Inferential Statistics:}
\item
  \textbf{Data Visualization:}
\item
  \textbf{Data Analysis:}
\item
  \textbf{Quantitative Research:}
\end{itemize}

\subsection*{Readings}\label{readings-14}
\addcontentsline{toc}{subsection}{Readings}

\subsubsection*{Required}\label{required-13}
\addcontentsline{toc}{subsubsection}{Required}

\begin{itemize}
\item
\item
\end{itemize}

\subsubsection*{\texorpdfstring{\emph{Optional}}{Optional}}\label{optional-13}
\addcontentsline{toc}{subsubsection}{\emph{Optional}}

\begin{itemize}
\item
\end{itemize}

\subsection*{Assignments}\label{assignments-14}
\addcontentsline{toc}{subsection}{Assignments}

\subsubsection*{Due Before Class}\label{due-before-class-12}
\addcontentsline{toc}{subsubsection}{Due Before Class}

\begin{itemize}
\item
\end{itemize}

\subsubsection*{\texorpdfstring{Due Before \emph{Next}
Class}{Due Before Next Class}}\label{due-before-next-class-13}
\addcontentsline{toc}{subsubsection}{Due Before \emph{Next} Class}

\begin{itemize}
\item
\end{itemize}

\section{Lecture-14: OLS Regression (Part
3)}\label{lecture-14-ols-regression-part-3}

View on Course Website

\subsection*{Topics}\label{topics-14}
\addcontentsline{toc}{subsection}{Topics}

\begin{itemize}
\tightlist
\item
  \textbf{Inferential Statistics:}
\item
  \textbf{Data Visualization:}
\item
  \textbf{Data Analysis:}
\item
  \textbf{Quantitative Research:}
\end{itemize}

\subsection*{Readings}\label{readings-15}
\addcontentsline{toc}{subsection}{Readings}

\subsubsection*{Required}\label{required-14}
\addcontentsline{toc}{subsubsection}{Required}

\begin{itemize}
\item
\item
\end{itemize}

\subsubsection*{\texorpdfstring{\emph{Optional}}{Optional}}\label{optional-14}
\addcontentsline{toc}{subsubsection}{\emph{Optional}}

\begin{itemize}
\item
\end{itemize}

\subsection*{Assignments}\label{assignments-15}
\addcontentsline{toc}{subsection}{Assignments}

\subsubsection*{Due Before Class}\label{due-before-class-13}
\addcontentsline{toc}{subsubsection}{Due Before Class}

\begin{itemize}
\item
\end{itemize}

\subsubsection*{\texorpdfstring{Due Before \emph{Next}
Class}{Due Before Next Class}}\label{due-before-next-class-14}
\addcontentsline{toc}{subsubsection}{Due Before \emph{Next} Class}

\begin{itemize}
\item
\end{itemize}

\section{Lecture-15: Analysis of
Variance}\label{lecture-15-analysis-of-variance}

View on Course Website

\subsection*{Topics}\label{topics-15}
\addcontentsline{toc}{subsection}{Topics}

\begin{itemize}
\tightlist
\item
  \textbf{Inferential Statistics:}
\item
  \textbf{Data Visualization:}
\item
  \textbf{Data Analysis:}
\item
  \textbf{Quantitative Research:}
\end{itemize}

\subsection*{Readings}\label{readings-16}
\addcontentsline{toc}{subsection}{Readings}

\subsubsection*{Required}\label{required-15}
\addcontentsline{toc}{subsubsection}{Required}

\begin{itemize}
\item
\item
\end{itemize}

\subsubsection*{\texorpdfstring{\emph{Optional}}{Optional}}\label{optional-15}
\addcontentsline{toc}{subsubsection}{\emph{Optional}}

\begin{itemize}
\item
\end{itemize}

\subsection*{Assignments}\label{assignments-16}
\addcontentsline{toc}{subsection}{Assignments}

\subsubsection*{Due Before Class}\label{due-before-class-14}
\addcontentsline{toc}{subsubsection}{Due Before Class}

\begin{itemize}
\item
\end{itemize}

\subsubsection*{\texorpdfstring{Due Before \emph{Next}
Class}{Due Before Next Class}}\label{due-before-next-class-15}
\addcontentsline{toc}{subsubsection}{Due Before \emph{Next} Class}

\begin{itemize}
\item
\end{itemize}

\section{Lecture-16: Chi-Squared}\label{lecture-16-chi-squared}

View on Course Website

\subsection*{Topics}\label{topics-16}
\addcontentsline{toc}{subsection}{Topics}

\begin{itemize}
\tightlist
\item
  \textbf{Inferential Statistics:}
\item
  \textbf{Data Visualization:}
\item
  \textbf{Data Analysis:}
\item
  \textbf{Quantitative Research:}
\end{itemize}

\subsection*{Readings}\label{readings-17}
\addcontentsline{toc}{subsection}{Readings}

\subsubsection*{Required}\label{required-16}
\addcontentsline{toc}{subsubsection}{Required}

\begin{itemize}
\item
\item
\end{itemize}

\subsubsection*{\texorpdfstring{\emph{Optional}}{Optional}}\label{optional-16}
\addcontentsline{toc}{subsubsection}{\emph{Optional}}

\begin{itemize}
\item
\end{itemize}

\subsection*{Assignments}\label{assignments-17}
\addcontentsline{toc}{subsection}{Assignments}

\subsubsection*{Due Before Class}\label{due-before-class-15}
\addcontentsline{toc}{subsubsection}{Due Before Class}

\begin{itemize}
\item
\end{itemize}

\subsubsection*{\texorpdfstring{Due Before \emph{Next}
Class}{Due Before Next Class}}\label{due-before-next-class-16}
\addcontentsline{toc}{subsubsection}{Due Before \emph{Next} Class}

\begin{itemize}
\item
\end{itemize}

\bibliography{book.bib,packages.bib}


\end{document}
