\documentclass[]{book}
\usepackage{lmodern}
\usepackage{amssymb,amsmath}
\usepackage{ifxetex,ifluatex}
\usepackage{fixltx2e} % provides \textsubscript
\ifnum 0\ifxetex 1\fi\ifluatex 1\fi=0 % if pdftex
  \usepackage[T1]{fontenc}
  \usepackage[utf8]{inputenc}
\else % if luatex or xelatex
  \ifxetex
    \usepackage{mathspec}
  \else
    \usepackage{fontspec}
  \fi
  \defaultfontfeatures{Ligatures=TeX,Scale=MatchLowercase}
\fi
% use upquote if available, for straight quotes in verbatim environments
\IfFileExists{upquote.sty}{\usepackage{upquote}}{}
% use microtype if available
\IfFileExists{microtype.sty}{%
\usepackage{microtype}
\UseMicrotypeSet[protrusion]{basicmath} % disable protrusion for tt fonts
}{}
\usepackage[margin=1in]{geometry}
\usepackage{hyperref}
\hypersetup{unicode=true,
            pdftitle={SOC 4015 \& 5050: Quantitative Analysis},
            pdfauthor={Christopher Prener, Ph.D.},
            pdfborder={0 0 0},
            breaklinks=true}
\urlstyle{same}  % don't use monospace font for urls
\usepackage{natbib}
\bibliographystyle{apalike}
\usepackage{longtable,booktabs}
\usepackage{graphicx,grffile}
\makeatletter
\def\maxwidth{\ifdim\Gin@nat@width>\linewidth\linewidth\else\Gin@nat@width\fi}
\def\maxheight{\ifdim\Gin@nat@height>\textheight\textheight\else\Gin@nat@height\fi}
\makeatother
% Scale images if necessary, so that they will not overflow the page
% margins by default, and it is still possible to overwrite the defaults
% using explicit options in \includegraphics[width, height, ...]{}
\setkeys{Gin}{width=\maxwidth,height=\maxheight,keepaspectratio}
\IfFileExists{parskip.sty}{%
\usepackage{parskip}
}{% else
\setlength{\parindent}{0pt}
\setlength{\parskip}{6pt plus 2pt minus 1pt}
}
\setlength{\emergencystretch}{3em}  % prevent overfull lines
\providecommand{\tightlist}{%
  \setlength{\itemsep}{0pt}\setlength{\parskip}{0pt}}
\setcounter{secnumdepth}{5}
% Redefines (sub)paragraphs to behave more like sections
\ifx\paragraph\undefined\else
\let\oldparagraph\paragraph
\renewcommand{\paragraph}[1]{\oldparagraph{#1}\mbox{}}
\fi
\ifx\subparagraph\undefined\else
\let\oldsubparagraph\subparagraph
\renewcommand{\subparagraph}[1]{\oldsubparagraph{#1}\mbox{}}
\fi

%%% Use protect on footnotes to avoid problems with footnotes in titles
\let\rmarkdownfootnote\footnote%
\def\footnote{\protect\rmarkdownfootnote}

%%% Change title format to be more compact
\usepackage{titling}

% Create subtitle command for use in maketitle
\newcommand{\subtitle}[1]{
  \posttitle{
    \begin{center}\large#1\end{center}
    }
}

\setlength{\droptitle}{-2em}
  \title{SOC 4015 \& 5050: Quantitative Analysis}
  \pretitle{\vspace{\droptitle}\centering\huge}
  \posttitle{\par}
  \author{Christopher Prener, Ph.D.}
  \preauthor{\centering\large\emph}
  \postauthor{\par}
  \predate{\centering\large\emph}
  \postdate{\par}
  \date{2018-04-22}

\usepackage{booktabs}
\usepackage{amsthm}
\makeatletter
\def\thm@space@setup{%
  \thm@preskip=8pt plus 2pt minus 4pt
  \thm@postskip=\thm@preskip
}
\makeatother

\newenvironment{rmdblock}[1]
  {\begin{shaded*}
  \begin{itemize}
  \renewcommand{\labelitemi}{
    \raisebox{-.7\height}[0pt][0pt]{
      {\setkeys{Gin}{width=3em,keepaspectratio}\includegraphics{images/#1}}
    }
  }
  \item
  }
  {
  \end{itemize}
  \end{shaded*}
  }
\newenvironment{rmdnote}
  {\begin{rmdblock}{note}}
  {\end{rmdblock}}
\newenvironment{rmdtip}
  {\begin{rmdblock}{tip}}
  {\end{rmdblock}}
\newenvironment{rmdwarning}
  {\begin{rmdblock}{warning}}
  {\end{rmdblock}}

\usepackage{amsthm}
\newtheorem{theorem}{Theorem}[chapter]
\newtheorem{lemma}{Lemma}[chapter]
\theoremstyle{definition}
\newtheorem{definition}{Definition}[chapter]
\newtheorem{corollary}{Corollary}[chapter]
\newtheorem{proposition}{Proposition}[chapter]
\theoremstyle{definition}
\newtheorem{example}{Example}[chapter]
\theoremstyle{definition}
\newtheorem{exercise}{Exercise}[chapter]
\theoremstyle{remark}
\newtheorem*{remark}{Remark}
\newtheorem*{solution}{Solution}
\begin{document}
\maketitle

{
\setcounter{tocdepth}{1}
\tableofcontents
}
\chapter*{Basics}\label{basics}
\addcontentsline{toc}{chapter}{Basics}

\begin{rmdwarning}
This is the \textbf{draft} syllabus for \textbf{Fall 2018}. Changes
should expected before the release of the syllabus prior to the first
day of class. This will occur in mid August.
\end{rmdwarning}

\subsection*{Course Meetings}\label{course-meetings}
\addcontentsline{toc}{subsection}{Course Meetings}

\emph{When:} Mondays, 4:15pm to 7:00pm

\emph{Where:} 3600 Morrissey (GeoSRI Lab)

\subsection*{Course Website}\label{course-website}
\addcontentsline{toc}{subsection}{Course Website}

\url{https://slu-soc5050.github.io}

\subsection*{Chris's Information}\label{chriss-information}
\addcontentsline{toc}{subsection}{Chris's Information}

\emph{Office:} 1918 Morrissey Hall

\emph{Email:}
\href{mailto:chris.prener@slu.edu}{\nolinkurl{chris.prener@slu.edu}}

\emph{GitHub:} \texttt{@chris-prener}

\emph{Slack:} \texttt{@chris}

\emph{Office Hours:}

\begin{itemize}
\item
  Mondays, 7:00pm to 7:30pm in 3600 Morrissey (GeoSRI Lab)
\item
  Wednesdays, 10:00am to 12:00pm in 3600 Morrissey (GeoSRI Lab)
\end{itemize}

\section*{License}\label{license}
\addcontentsline{toc}{section}{License}

Copyright © 2016-2018 \href{https://chris-prener.github.io}{Christopher
G. Prener}

This work is licensed under a Creative Commons Attribution-ShareAlike
4.0 International License.

\part{Syllabus}\label{part-syllabus}

\chapter{Course Introduction}\label{course-introduction}

\begin{quote}
Figures often beguile me, particularly when I have the arranging of them
myself; in which case the remark attributed to Disraeli would often
apply with justice and force: `There are three kinds of lies: lies,
damned lies, and statistics'.
\end{quote}

\textbf{Mark Twain (1906)}

This course provides an introduction to applied statistical analysis for
both undergraduate and graduate students with an emphasis placed on
statistical techniques that are most common in the sociological
literature. The statistical techniques introduced include measures of
central tendency and dispersion as well as measures of bi-variate
association. Multivariate statistical analyses are also introduced along
with essential skills for cleaning data, creating plots, and reporting
results. While the examples may be specific to the social sciences, the
theories and skills that are covered are broadly applicable across
academic disciplines.

\section{Course Objectives}\label{course-objectives}

This course has four intertwined objectives. After completing the
course, students will be able to:

\begin{enumerate}
\def\labelenumi{\arabic{enumi}.}
\item
  \emph{Fundamentals of inferential statistics}: Describe the use of
  various statistical tests, their requirements and assumptions, and
  their interpretation; execute these tests both by hand and
  programmatically using \texttt{R}.
\item
  \emph{Fundamentals of data analysis}: Perform basic data cleaning and
  analysis tasks programmatically using \texttt{R} in ways that support
  high quality documentation and replication.
\item
  \emph{Fundamentals of data visualization}: Create and present
  publication quality plots programmatically using \texttt{R} and
  \texttt{ggplot2}.
\item
  \emph{Quantitative research synthesis}: Plan, implement (using
  \texttt{R}), and present (using LaTeX and the presentation software of
  your choice) a research project that uses linear regression to answer
  a research question.
\end{enumerate}

\section{Core Resources}\label{core-resources}

There are three core documents and resources for this course. This
\textbf{Syllabus} sets out core expectations and policies for the
course. It includes a \textbf{Reading List} contains topics, readings
(both required and optional), and assignment due dates for each week.
These two documents spell out what is \emph{required} for this course.
Once the semester starts, these documents will only be updated if a
schedule change is necessary.

In addition to these documents, regular updates will be provided on the
\href{https://slu-soc5050.github.io}{\textbf{course website}}. Each week
will have a corresponding page on the site that includes links to
handouts, YouTube videos, sample code, and additional descriptions of
concepts covered in class. Please check the website regularly for
updates and new content.

All of our course content will be hosted through GitHub, where we have a
\href{https://github.com/slu-soc5050}{\textbf{course organization}}. The
course organization will contain \emph{repositories} (a special type of
folder) for each lecture as well as several repositories dedicated to
specific aspects of the course. These repositories should be cloned as
they are released and synced as they are updated.

\bibliography{book.bib,packages.bib}


\end{document}
